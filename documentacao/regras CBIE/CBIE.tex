\documentclass[a4paper,10pt]{article}
\usepackage[utf8]{inputenc}
\usepackage[brazil]{babel}
\usepackage[T1]{fontenc}

\begin{document}
\begin{flushleft}
\large\textbf{CRITÉRIOS CBIE 2012 - ARTIGO\\[0.5cm]}
\end{flushleft}

\begin{flushleft}
Para a submissão, devemos nos guiar por três questões:\\
1 - Qual é a contribuição em termos tecnológicos?\\
2 - Qual é a contribuição para a aprendizagem?\\
3 - Qual é a metodologia científica empregada no trabalho?\\
\end{flushleft}

\begin{flushleft}
\textbf{ABSTRACT / RESUMO}
\end{flushleft}

Neste ponto é solicitada a escrita de "resumo e abstrat" apenas para os artigos escritos em português. Artigos em Inglês ou Espanhol deverão conter apenas o resumo na língua desejada. Em todos os casos, o autor deve tomar cuidado para que o resumo não ultrapassem 10 linhas cada, sendo que o mesmo deverá estar na primeira página.\\  

\begin{flushleft}
\textbf{INFORMAÇÕES GERAIS}
\end{flushleft}

Artigos escritos em Inglês, português ou espanhol. O formato de papel A4, com coluna única, -- "3,5 cm para margem superior" --  "2,5 centímetro de margem inferior" -- e "3,0 cm para as margens laterais", sem cabeçalhos e rodapés. A fonte principal deve ser Times, tamanho nominal de 12pt, com 6pt de espaço antes de cada parágrafo. Números de página deve ser suprimida.\\

\begin{flushleft}
\textbf{PRIMEIRA PÁGINA}
\end{flushleft}

O título do trabalho deve ser centralizado sobre a página inteira, com 16pt em negrito e com 12pt de espaço antes de si mesmo; Os nomes dos autores também centralizados em fonte de 12pt, em negrito, todos dispostos na mesma linha, separados por vírgulas e com 12pt do espaço após o título; Endereço dos autores devem estar também centralizados em fonte 12pt, também com 12pt do espaço após os nomes dos autores; Os endereços de e-mail escritos usando a fonte Courier New, tamanho nominal de 10pt, com 6pt de espaço antes e 6pt de espaço depois. Abstract e resumo devem estar em fonte Times 12pt, recuado 0,8 centímetros em ambos os lados. A palavra "Abstract e Resumo", devem estar escritas em negrito e preceder o texto.\\

\begin{flushleft}
\textbf{SEÇÕES E PARÁGRAFOS}
\end{flushleft}

Os títulos das seções em negrito, 13pt, alinhado à esquerda. Com um extra de 12pt de espaço antes de cada título. Seção numeração é opcional. O primeiro parágrafo de cada seção não deve ser recuado, enquanto que as primeiras linhas dos parágrafos subsequentes deve ser recuado por 1,27 centímetros. Subseções: Os títulos das subseções devem estar em negrito, 12pt, alinhado à esquerda.\\

\pagebreak

\begin{flushleft}
\textbf{FIGURAS E LEGENDAS}
\end{flushleft}

Legendas das figuras e tabelas deve ser centralizado em menos de uma linha, caso contrário, justificada e recuada por 0,8 centímetros em ambas as margens. A fonte da legenda deve ser Helvetica, 10pt, negrito, com 6pt de espaço antes e depois de cada legenda. Nas tabelas, evitar o uso de fundos coloridos ou sombreados, espessuras ou linhas de enquadramento desnecessários. Ao relatar dados empíricos, não use mais dígitos decimais do que justifiquem sua precisão e reprodutibilidade. A legenda da tabela deve ser colocada antes da mesma e a fonte usada deve também ser Helvetica, 10pt, negrito, com 6pt de espaço antes e depois de cada legenda.\\ 

\begin{flushleft}
\textbf{IMAGENS}
\end{flushleft}

Todas as imagens e ilustrações devem estar em tons de preto e branco ou cinza, com exceção para os trabalhos que estarão disponíveis eletronicamente (em CD-ROMs, internet, etc.) A resolução da imagem em papel deve ser de cerca de 600 dpi para imagens em preto-e-branco, e 150-300 dpi para imagens em tons de cinza. Não inclua imagens com resolução excessiva, pois podem levar horas para imprimir, sem nenhuma diferença visível no resultado.\\

\begin{flushleft}
\textbf{REFERÊNCIAS}
\end{flushleft}

As referências bibliográficas devem ser inequívocas e uniformes. É recomendado dar os nomes de autores e referências entre colchetes, por exemplo, [Knuth 1984], [Boulic e Renault 1991], ou datas entre parênteses, por exemplo, Knuth (1984), Smith e Jones (1999). As referências devem ser listadas usando fonte tamanho 12pt, com 6pt de espaço antes de cada referência. A primeira linha de cada referência não deve estar recuada, enquanto o posterior deve ser recuado em 0,5 cm.\\

 \end{document}
