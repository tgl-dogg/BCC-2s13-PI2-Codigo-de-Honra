\documentclass{sciposter}
\usepackage{lipsum}
\usepackage{epsfig}
\usepackage{amsmath}
\usepackage{amssymb}
\usepackage{multicol}
\usepackage{graphicx,url}
\usepackage[portuges, brazil]{babel}   
\usepackage[utf8]{inputenc}
%\usepackage{fancybullets}
\newtheorem{Def}{Definition}


\title{Projeto Integrador II\\ Código de Honra}
%Título do projeto

\author{Caroline Bomfim, Gabriel Garcia, Paulo Henrique Leite, Vinicius de Carvalho}
%nome dos autores

\institute 
{Bacharelado em Ciência da Computação\\
Centro Universitário SENAC - Campus Santo Amaro
  (SENAC-SP)\\
  Av. Engenheiro Eusébio Stevaux, 823 -- Santo Amaro, São Paulo -- CEP 04696-000 -- SP -- Brasil}
%Nome e endereço da Instituição

\email{{caroline.bomfim@hotmail.com.br, gabrielgfa1@gmail.com, paulo-http@live.com, vinidoggll@hotmail.com},{(@gmail.com})}
% Onde você coloca os emails dos integrantes


%\date is unused by the current \maketitle

\rightlogo[1]{Senac-logo}
\leftlogo[1]{Ecomp-logo}
% Exibe os logos (direita e esquerda) 
% Procure usar arquivos png ou jpg, e de preferencia mantenha na mesma pasta do .tex
%%%%%%%%%%%%%%%%%%%%%%%%%%%%%%%%%%%%%%%%%%%%%%%%%%%%%%%%%%%%%%%%%%%%%%%%%%%%%%%%
%%% Begin of Document



\begin{document}
%define conference poster is presented at (appears as footer)

\conference{{\bf E-COMP 2013}, 5º Encontro da Computação - Senac, 27 de Novembro de 2013, São Paulo, Brasil}

%\LEFTSIDEfootlogo  
% Uncomment to put footer logo on left side, and 
% conference name on right side of footer

% Some examples of caption control (remove % to check result)

%\renewcommand{\algorithmname}{Algoritme} % for Dutch

%\renewcommand{\mastercapstartstyle}[1]{\textit{\textbf{#1}}}
%\renewcommand{\algcapstartstyle}[1]{\textsc{\textbf{#1}}}
%\renewcommand{\algcapbodystyle}{\bfseries}
%\renewcommand{\thealgorithm}{\Roman{algorithm}}

\maketitle

%%% Begin of Multicols-Enviroment
\begin{multicols}{3}

%%% Abstract
\begin{abstract}
Como proposto no Projeto Integrador II, durante o semestre foi desenvolvido um jogo educativo, com o uso da biblioteca Allegro 5, em C. O jogo consiste do ensino de Algoritmos e Programação 1.
\end{abstract}

%%% Introduction
\section{Introducão}
Baseando-se na diculdade interna e externa do grupo, considerando também pessoas de diferetes áreas, verica-se que é possivel melhorar o desempenho
com a programação a partir de um primeiro contato mais amigável. Isso pode
despertar interesse em pessoas não ligadas a área da computação e não cria
uma dependencia de linguagem, podendo assim desenvolver-se em várias outras
linguagens com maior facilidade, de maneira divertida e interativa.\\		
 





\section{Objetivos}
Diversos estudos comprovam que os jogos passam uma série de aprendizados
aos seus jogadores, sejam eles bons ou ruins. Devido a atenção e o foco que
os jogos obtém de seus jogadores, principalmente em jogos que contém desaos
e entreterimento lógico, acabam passando algum conhecimento, porém, despreocupado
onde isso seja efetivamente utilizado.\\
Em fevereiro de 2013, duas das maiores empresas do ramo de informática
tiveram a visão de que os jogos podem ensinar coisas muito úteis, desde que
sejam focados em tal objetivo. A Microsoft Corporation e a Facebook Inc. se
uniram para criar jogos que pudessem ensinar programação, baseado no fato
de que esta é uma área em ascenção e que se torna imprescindível o ensino de
lógica de programação para as gerações futuras, onde softwares não são apenas
uma tendência, mas uma realidade.\\
Não sendo os únicos nesta empreitada, a ideia de se ensinar lógica de programa
ção de modo interativo já vinha sido trabalhada com a empresa Code.org, no
desenvolvimento do Codecademy, que abrange diversas linguagens, entre elas,
Python, Ruby, PHP, HTML e JavaScript, e uma versão mais infantil conhecida
como Scratch, jogo capaz de ensinar crianças os primeiros passos da arte de
programar.\\
Visualizando esta oportunidade, o grupo almeja o desenvolvimento de um
jogo que possa ensinar o desenvolvimento de algoritmos de modo simples, permitindo
um primeiro contato com a área de forma amigável e o mais independente
possível de linguagens de programação, criando uma estrutura de conhecimento
que possa ser expandida posteriormente de acordo com o interesse e
necessidade de cada usuário.\\



\section{Metodologia}
Usamos como metodologia, a divisão da matéria em 3. Condicionais, Repetições, e Vetores. E cada uma dessas divisões, ter uma fase própria, e a última delas, um compilado de todas. Meio que um "boss".\\
Tentamos implementar a imersão do usuário numa história, que ele deveria resolver problemas por meio de algoritmos dados por ele, que com a combinação correta, levaria ele para a próxima fase. Para aprender mais.\\
Com isso, com o usuário passando pela fase, podemos assegurar que ele está pronto, e tem o conhecimento necessário para poder aprender mais sobre Algoritmos.\\

\section{Resultados e Discussão}

Verificar os principais resultados obtidos de acordo com os objetivos propostos.\\

Nacken \cite{Nacken:thesis} derived an algorithm for computation
of pattern spectra for granulometries based on openings by discs of increasing
radius for various metrics, using the opening transform. After the
opening transform has been computed, it is straightforward to compute the 
pattern spectrum:
\begin{itemize}
\item Set all elements of array {\tt S} to zero
\item For all $x \in X$ increment {\tt S}[$\Omega_X(x)$] by one. 
\end{itemize}

To compute the pattern \emph{moment} spectrum, the only thing that needs to be
changed is the way {\tt S}[$\Omega_X(x)$] is incremented. As shown in Algorithm
\ref{alg:spect}.

\begin{algorithm}
\begin{itemize}
\item Set all elements of array {\tt S} to zero
\item For all $(x,y) \in X$ increment {\tt S}[$\Omega_X(x,y)$] by 
$x^iy^j$. 
\end{itemize}
\caption{ Algorithm for computation of pattern moment
spectrum of order $ij$. \label{alg:spect}}
\end{algorithm}

This algorithm can 
readily be adapted to other granulometries, simply by computing the 
appropriate opening transform.



\section{Conclusão}

Sitting on a corner all alone,
staring from the bottom of his soul,
watching the night come in from the window
\\
It'll all collapse tonight, the fullmoon is here again
In sickness and in health, understanding so demanding
It has no name, there's one for every season
Makes him insane to know
\\
Running away from it all
"I'll be safe in the cornfields", he thinks
Hunted by his own,
again he feels the moon rising on the sky
\\
Find a barn which to sleep in, but can he hide anymore
Someone's at the door, understanding too demanding
Can this be wrong, it's love that is not ending
Makes him insane to know
\\
She should not lock the open door
(Run away, run away, run away)
Fullmoon is on the sky and He's not a man anymore
sees the change in him but can't
(Run away, run away, run away)
See what became out of her man
Fullmoon
\\
Swimming across the bay,
the night is gray, so calm today
She doesn't wanna wait.
"We've gotta make the love complete tonight..."
\\
In the mist of the morning he cannot fight anymore
Hundred moons or more, he's been howling
Knock on the door, and scream that is soon ending
Mess on the floor again
\\
She should not lock the open door
(Run away, run away, run away)
Fullmoon is on the sky and he's not a man anymore
She sees the changes in him but can't
(Run away, run away, run away)
See what became out of her man
\\
She should not lock the open door
(Run away, run away, run away)
Fullmoon is on the sky and he's not a man anymore
sees the changes in him but can't
(Run away, run away, run away)
See what became out of her darling man
\\
She should not lock the open door
(Run away, run away, run away)
Fullmoon is on the sky and he's not a man anymore
See what became out of her man
 
%%% References

%% Note: use of BibTeX als works!!

\bibliographystyle{plain}
\begin{thebibliography}{1}

\bibitem{Wikipedia}
\newblock http://pt.wikipedia.org/wiki/HistC3%B3ria_dos_jogos_eletr%C3%B4nicos
\newblock {\em História dos Jogos Eletrônicos}, 20/08/2013.

\bibitem{Terra}
\newblock http://tecnologia.terra.com.br/negocios-e-ti/aplicativos-ensinam-programacaoa-
criancas-conheca,db17bf4f0b32e310VgnVCM\\
4000009bcceb0aRCRD.html
\newblock {\em Aplicativos ensinam programação}, 20/08/2013.

\bibitem{Gizmodo}
\newblock http://gizmodo.uol.com.br/os-quatro-gigantes-da-tecnologia-que-nao-acabarama-
faculdade/\\
\newblock {\em Os Quatro Gigantes da tecnologia que não acabaram.}, 20/08/2013

\bibitem{Terra}
\newblock http://tecnologia.terra.com.br/negocios-e-ti/zuckerberg-e-bill-gates-incentivamensino-
de-programacao,114106b14481d310VgnVCM\\
20000099cceb0aRCRD.html
\newblock {\em Zukerberg e Gates incentivam ensino de programação}, 20/08/2013

\end{thebibliography}

\end{multicols}

\end{document}