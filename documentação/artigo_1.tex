
\documentclass[a4paper,10pt]{article}
\usepackage[brazil]{babel}
\usepackage[latin1]{inputenc}
\usepackage[T1]{fontenc}

\begin{document}
\begin{center}
\Large\textbf{COMPOSI��O DO GRUPO}
\end{center}
\textbf{Nome:} Caroline Bomfim do Espirito Santo\\
\textbf{Email:} caroline.bomfim@hotmail.com.br\\
\textbf{Nome:} Gabriel Garcia Ferraz do Amaral\\
\textbf{Email:} gabrielgfa1@gmail.com\\
\textbf{Nome:} Paulo Henrique Fernandes Leite\\
\textbf{Email:} paulo-http@live.com\\
\textbf{Nome:} Vinicius de Carvalho\\
\textbf{Email:} vinidoggll@hotmail.com\\
\\
\large\textbf{TEMA A SER TRABALHADO}\\
\\
Introduzir o usu�rio em conceitos de programa��o e pensamento l�gico por meio de um jogo educativo baseado em sintaxe de linguagem C, que possui comandos de taxonomia n�o muito espec�ficas, permitindo posteriormente um contato com a programa��o o mais independente poss�vel de linguagem. 
\\
\\
\large\textbf{JUSTIFICANDO O TEMA}\\
\\
Baseando-se na dificuldade interna e externa do grupo, considerando tamb�m pessoas de diferetes �reas, verifica-se que � possivel melhorar o desempenho com a programa��o a partir de um primeiro contato mais amig�vel. Isso pode despertar interesse em pessoas n�o ligadas a �rea da computa��o e n�o cria uma dependencia de linguagem, podendo assim desenvolver-se em v�rias outras linguagens com maior facilidade, de maneira divertida e interativa.
\\
\\
\large\textbf{MOTIVA��O}\\
\\
Em fevereiro de 2013, dois gigantes da tecnologia, Bill Gates, da empresa Microsoft e Mark Zuckerberg, criador do Facebook, se uniram com a id�ia de criar aplicativos em que ensinem crian�as a programar. Dados indicam que isso pode alavancar o interese das pessoas nas �reas de programa��o, podendo at� mesmo instimular o governo a implantar a id�ia em seu plano de educa��o. Com base nisto, o grupo visualizou uma oportunidade de agregar o conhecimento as pessoas com uma aplica��o mais b�sica e introdut�ria, entretanto educativa, e que se possivel, n�o s� adicione conhecimento mas tamb�m interesse pela �rea.   
\end{document}